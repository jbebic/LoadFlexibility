\documentclass[11pt]{article}
\usepackage{lipsum}
\usepackage[letterpaper, landscape, twocolumn, left=0.5in, right=0.5in, top=1.25in, bottom=0.75in, headheight=0.6in]{geometry} % controls page layout
\usepackage{fancyhdr} % controls headers and footers
\usepackage{lastpage} % enables # of ##
\usepackage{graphicx} % graphics
\usepackage{grffile} % enables multiple periods in file name of \includegraphics
\usepackage{amsmath}  % extended mathematics
\usepackage{booktabs} % book-quality tables
\usepackage{units}    % non-stacked fractions and better unit spacing
\usepackage[T1]{fontenc} % 8-bit encoding of fonts (glyphs are a part of fonts)
% \usepackage{siunitx} % table alignment
\usepackage{etoolbox} % conditional compilation

\fancypagestyle{title}{ % defining header style for the first page
  \fancyhead[C,R]{}
  \fancyhead[L]{\begin{tabular}{ll}
  \raisebox{-.47\height}{\includegraphics[height=0.75in, trim=0 -0.1in 0 0]{common/SCE_logo}} &
  \textsc{\textbf{\Huge Customer Report for: <CustomerID>}}
  \end{tabular}}
  \fancyfoot[L]{\today}
  \fancyfoot[C]{}
  \fancyfoot[R]{\thepage/\pageref{LastPage}}
}

\fancypagestyle{energy}{ % header style for energy charges
  \fancyhead[L]{\textsc{\textbf{\Huge Energy Charges}}}
  \fancyhead[C]{}
  \fancyhead[R]{\textsc{\textbf{\Huge <CustomerID>}}}
  \fancyfoot[L]{\today}
  \fancyfoot[C]{}
  \fancyfoot[R]{\thepage/\pageref{LastPage}}
}

\fancypagestyle{demand}{ % header style for demand charges
  \fancyhead[L]{\textsc{\textbf{\Huge Demand Charges}}}
  \fancyhead[C]{}
  \fancyhead[R]{\textsc{\textbf{\Huge <CustomerID>}}}
  \fancyfoot[L]{\today}
  \fancyfoot[C]{}
  \fancyfoot[R]{\thepage/\pageref{LastPage}}
}

\renewcommand{\headrulewidth}{0.4pt}
\renewcommand{\footrulewidth}{0.4pt}

\begin{document}

%% Introductory Page
\pagestyle{title}
This report summarizes the time-dependent portion of your electricity bill, the portion of the charges dependent on your usage patterns.
It also compares your charges relative to those of others in your ``peer group'' -- those within the same North American Industry (NAICS) code and subscribed to the same time of use rate (TOU).
It is provided to you for your information and as a backdrop for the conversation with Southern California Edison (SCE) to find opportunities for savings.

The analysis is performed by back-casting your last year's consumption to the rates currently in effect.
As such, it is a forecast of your future payments, not an accurate replica of your earlier ones.

\vspace{3ex}
\textbf{\Large Components of the Bill}
\vspace{1ex}

The time-dependent part of an electric bill has three main components: energy charges, facility charges, and demand charges.
The makeup of these charges for the year and by month are shown in Figures~\ref{fig:pie} and \ref{fig:bars}, respectively.
\begin{figure}[!h]
\centering
\includegraphics[height=1.8in, page=1, trim=0in 0.45in 0in 0.45in, clip]{visuals/<CustomerID>.piechart.pdf}
\caption{Makeup of annual charges}
\label{fig:pie}
\end{figure}

\begin{figure}[!h]
\centering
\includegraphics[width=0.9\columnwidth, page=1, trim=0in 0.45in 0in 0.45in, clip]{visuals/<CustomerID>.boxplot.pdf}
\caption{Makeup of monthly charges}
\label{fig:bars}
\end{figure}

The first component, \emph{energy}, is the easiest to understand.
Those charges are a reflection of the monthly energy you consumed, metered in 15~min intervals, and priced at the time-varying rates.
The other two, \emph{facility} and \emph{demand}, are different.
They represent charges due to the shape of your load profile -- the more uneven it gets, the more you pay.

In your case <dmndpc>\% of the total annual bill is determined by the shape of your load, not by the energy you consumed.
It is therefore possible to reduce your electric bill by changing the shape of your consumption.
% Improvements are often possible by implementing changes that do not interfere with your operations.
We are taking this initiative to determine what is possible, at what cost, and if our customers are receptive to such changes.
For reference, the annual charges by bill component are also summarized in Table~\ref{tab:annual}.

\begin{table}[th!]
  \centering
  \caption{Components of Annual Electric Bill}
  \vspace{1.5ex}
  \label{tab:annual}
  \begin{tabular}{lr}
    Bill component & Amount [\$] \\
    \midrule
    Energy & <energy> \\
    Facility & <facility> \\
    Demand & <demand> \\
    \midrule
    Total & <total>
  \end{tabular}
\end{table}
\clearpage

%% Energy Charges Page
\newgeometry{twocolumn, left=0.5in, right=0.5in, top=0.75in, bottom=0.75in, headheight=0.1in}
\pagestyle{energy}

We review energy charges first.
Your normalized consumption for the year is shown in Figure~\ref{fig:heatmap}.
\emph{Normalized} here means that the consumption metered in each interval is divided by the annual average.
This enables comparison of differently sized loads using the same color scale.
\begin{figure}[!h]
\centering
\includegraphics[width=\columnwidth, page=1, trim=0in 0.45in 0in 0.45in, clip]{visuals/<CustomerID>.heatmap.pdf}
\caption{Normalized consumption for <CustomerID>}
\label{fig:heatmap}
\end{figure}

The top part of the figure is the heatmap -- the days of the year are shown on the x-axis, the hours of the day on the y axis.
The bottom part of the figure show the statistical spread of the weekly observations shown as whisker (or box) plots.

The price of energy for your rate is shown in Figure~\ref{fig:toumap}.
\begin{figure}[!h]
\centering
\includegraphics[width=\columnwidth, page=1, trim=0in 3in 0in 0.45in, clip]{visuals/<CustomerID>.billing.Heatmap.pdf}
\caption{Energy price on <RateCode>}
\label{fig:toumap}
\end{figure}
The rate varies with the season of the year, weekdays versus weekends and public holidays, and with the time of day.
Note the darker vertical stripes denoting the weekends, and the high energy price during the summer-peak periods.

% \lipsum[1][1-7]
For reference, the monthly energy charges are summarized in Table~\ref{tab:energy}.
These charges can be reduced by shifting the load away from the more expensive periods.
% Techniques such as pre-cooling of the space before the peak rate comes into effect and letting the temperature get closer to the comfort limit during the peak rate are simple and effective ways to influence consumption and reduce costs.
Pre-cooling of the space before, and setting the thermostats closer to the comfort limits during the peak rate are simple and effective ways to influence consumption and reduce costs.

\begin{table}[th!]
  \centering
  \caption{Monthly Energy Charges}
  \vspace{1.5ex}
  \label{tab:energy}
  \begin{tabular}{p{0.75in}rp{0.2in}p{0.75in}r}
    Month & Charge [\$] & & Month & Charge [\$] \\
    \midrule
    Jan & <EnAvg01> & & Jul & <EnAvg07> \\
    Feb & <EnAvg02> & & Aug & <EnAvg08> \\
    Mar & <EnAvg03> & & Sep & <EnAvg09> \\
    Apr & <EnAvg04> & & Oct & <EnAvg10> \\
    May & <EnAvg05> & & Nov & <EnAvg11> \\
    Jun & <EnAvg06> & & Dec & <EnAvg12> \\
    \midrule
    Average & <EnAvgAnnual> & & &
  \end{tabular}
\end{table}

\vspace{3ex}
\textbf{\Large Peer performance}
\vspace{1ex}

% \lipsum[1][1-7]
Your average energy price relative to the price paid by the members of your peer group is shown in Figure~\ref{fig:PeerCompEn}.

\begin{figure}[!h]
\centering
\includegraphics[width=\columnwidth, page=1, trim=0in 2.1in 0in 2.1in, clip]{visuals/<CustomerID>.whiskerchart.pdf}
\caption{Average annual energy price for <CustomerID> compared to peers}
\label{fig:PeerCompEn}
\end{figure}

\clearpage

%% Demand Charges Page
\newgeometry{twocolumn, left=0.5in, right=0.5in, top=0.75in, bottom=0.75in, headheight=0.1in}
\pagestyle{demand}
% \lipsum[1][1-7]

As was discussed earlier in the report, the demand charges are a function of the load shape.
Specifically, the function of the peak load in the rate-defined time intervals.
To understand the entitlement for reduction of demand charges it is therefore important to understand the load shapes.
Figure~\ref{fig:duration} shows your load shapes.
\begin{figure}[!h]
\centering
\includegraphics[width=\columnwidth, page=1, trim=0in 0.45in 0in 0.45in, clip]{visuals/<CustomerID>.duration.monthly.test.pdf}
\caption{Normalized load and demand profiles for <CustomerID>}
\label{fig:duration}
\end{figure}

The top half of the figure shows daily load profiles for each month of the year.
The bottom half shows the daily duration curves, obtained by sorting the 15~min load data in the descending order.
These curves provides insight into  the load peaks.
Clearly, sharp narrow peaks are easier to mitigate.
Further analysis can now be undertaken to understand which loads are contributing to the peaks and then scheduling their operation to prevent peak forming.
% \lipsum[1][1-7]

\vfill\eject % An equivalent of \columnbreak w/o using the multicolumn package

For reference, your monthly demand charges are summarized in Table~\ref{tab:demand}.
\begin{table}[th!]
  \centering
  \caption{Monthly Demand Charges}
  \vspace{1.5ex}
  \label{tab:demand}
  \begin{tabular}{p{0.75in}rp{0.2in}p{0.75in}r}
    Month & Charge [\$] & & Month & Charge [\$] \\
    \midrule
    Jan & <DmndAvg01> & & Jul & <DmndAvg07> \\
    Feb & <DmndAvg02> & & Aug & <DmndAvg08> \\
    Mar & <DmndAvg03> & & Sep & <DmndAvg09> \\
    Apr & <DmndAvg04> & & Oct & <DmndAvg10> \\
    May & <DmndAvg05> & & Nov & <DmndAvg11> \\
    Jun & <DmndAvg06> & & Dec & <DmndAvg12> \\
    \midrule
    Average & <DmndAvgAnnual>
  \end{tabular}
\end{table}

\vspace{3ex}
\textbf{\Large Peer performance}
\vspace{1ex}

Your average annual demand price relative to the price paid by the members of your peer group is shown in Figure~\ref{fig:PeerCompDmnd}.
%\lipsum[1][1-7]

\begin{figure}[!h]
\centering
\includegraphics[width=\columnwidth, page=1, trim=0in 3.8in 0in 0.5in, clip]{visuals/<CustomerID>.whiskerchart.pdf}
\caption{Average annual demand price for <CustomerID> compared to peers}
\label{fig:PeerCompDmnd}
\end{figure}

\vspace{3ex}
\textbf{\Large Opportunities for savings}
\vspace{1ex}

Your total annual consumption is <totalkWh>~kWh.
Considering Figures~\ref{fig:PeerCompEn} and \ref{fig:PeerCompDmnd} enables making an estimate of your savings per kWh by reducing your payments for energy and demand.
To project annual savings, multiply your total energy consumption with the differences estimated from Figures~~\ref{fig:PeerCompEn} and \ref{fig:PeerCompDmnd}.
% Percentiles relative to peer group:
% Energy ..... : <pcEn>
% Facility ... : <pcFac>
% Demand ..... : <pcDmnd>
% FnD ........ : <pcDnF>
% Total bill .... : <total>
% Projected savings
% Energy ..... : <sEn>
% Facility ... : <sDm>
% Demand ..... : <sFac>
% DnF ........ : <sDnF>

\end{document}
